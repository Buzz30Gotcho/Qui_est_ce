\section{Étape 2 : aider  à la saisie  des personnages}
\subsection{Description du problème: format des données et du résultat}
\begin{}
    Premièrement, au début de la programmation, nous avions eu deux choix de concept pour l'attribution des caractéristiques aux personnages:
        - Saisir manuellement (au clavier) chaque attributs, valeurs de chaque personnages
        - Créer les caracterisitiques, puis les attribuer aux personnages en les selectionnant 

    Les inconvénients et les avantages des deux concept : 
        - Pour la saisie manuelle :
            avantages: 
                \__ Développement d'apparence plus simple (peu de fonction à implémenter)
            inconvénients:
                \__ Malgré la possibilité de vérifier les occurences par comparaison orthographique, un "non" != "Non" != "Nope"
                    donc possibilité d'occurence
                \__ Dans le cas d'une envie de modifier ou des supprrimer un attrbuts ou une valeur, il faudrait apporter 
                    la modif manuellement pour chaque personnage
                \__ Complexifie les methodes de vérification de la validité des données
                \__ Saisie inconfortable des caracterisitiques

        - Pour la saisie des caracterisitiques et leur attribution aux personnage par simple selection :
            avantages: 
                \__ Vérification de la validité des données plus simple (unicité des caracterisitiques des personnages)
                \__ Modification et de suppression des caracterisitiques plus confortable et rapide
                    \___ possibilité d'attribuer automatiquement les attributs déjà créés aux personnages
                    \___ lors de la modif ou de la suppression d'un attrbut ou d'une valeur, chaque personnage ayant cet attribut
                         ou cette valeur recevra une modification automatique au niveau de sa liste de caractérisitque 
            
            inconvénients:
                \__ Malgré la possibilité de vérifier les occurences par comparaison orthographique, un "non" != "Non" != "Nope"
                    donc possibilité d'occurence
                \__ Conplexification du développement 
                    \__ Création d'une class de type fenetre pour la création et la modification des caracterisitiques
                    \__ Création d'une class de type fenetre pour l'attribution des valeurs aux attributs des personnages
                
    Après reflexion, nous avions alors opté pour le deuxième concept. 
    
    Deuxièmement, après s'être mis d'accord sur la manière de saisie des données, il n'y avait plus qu'à réfléchir 
sur la manière de recupérer les données de chaque personnage. 
Comme les paramètres de configuration de l'interface et le nom du plateau du qui-es-ce sont comptenus directement 
dans les attributs du générateur, il à été facile de les récolter. 
Cependant, les personnages sont des objets de class et que leurs données ont été placées dans leurs attributs, nous avions 
alors implementé une fonction parcourant les personnages un à un avec leurs attributs. Au fur et à mesure, la fonction 
récolter alors leurs données sous la forme d'une String, et lorsque tout les personnages ont été parcouru,
les données des personnages et de la configuration de l'interface du qui-es-ce sont écrites dans un fichier Json 
généré automatiquement et dont le nom est donnée en fonction du nom du dossier comptenant les images des différents personnages.
\end{}
\subsection{Scénario des interactions avec l'utilisateur}
\begin{}
    Excecution du programme : 
        (sans sauvegarde) code d'execution et chemin vers les images
        (avec sauvegarde) code d'execution et argument -save

    Initialisation de l'interface :
        Saisie des dimentions du plateau 
        Saisie de la taille des images representant les personnages
        Espacement entre les images

         /!\ messages d'erreur si dimention trop grande, taille d'image trop grande, espacement trop grand
    
    Initialisation des caracterisitiques : 
        Création / modification / suppression des attributs et des valeurs
            /!\ affichage d'erreur si création d'un attribut déjà existant
            /!\ affichage erreur si la modification du nom d'un attribut correspond à un autre
            /!\ confirmation si modification et avertissement si suppression d'une valeur

    Aperçu du plateur ; 

    Caractérisation des personnages du plateau :
        Les personnages possèdant des attributs grace à l'étape 2, on leur attribut des valeurs à chaqu'un de leur attributs
            /!\ affichage d'erreur si le set d'attribut du personnage n'est pas unique

    Verification finale du plateau :
        (auto) Verifi si toutes les caracteristiques de chaque personnage ont été attribuées d'une ou de plusieurs valeurs 
        /!\ message d'erreur si des personnages ont été mal caractérisés (coloration en rouge / vert si tout va bien)
        Si tout a été correctement fait, affichage d'un message indiquant la génération du fichier .json
\end{}
