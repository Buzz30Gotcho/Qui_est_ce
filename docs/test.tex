\documentclass{article}

\title{Projet S3 - Qui-est-ce ?}
\author{Laurent Antoinette, Romain Campillo, Tony Nguyen, Frederic Sar}

\pagenumbering{arabic}
\usepackage{graphicx}
\graphicspath{ {/home/tony/Images/} }

\begin{document}
\maketitle

\tableofcontents

\section{Introduction}
This is where you tell people why they should bother reading your article.
-projet s4
-4 personnes
-Qui-est-ce + générateur + rapport + présentation oral/démonstration
-3h/semaine en td
-moins de 3 mois pour le code
-CDIO, V life cycle 


\section{Literature Review}
This is the section that is invariably much longer than it should be, and
where everyone tries to impress peers about how easy it is to locate various
references in online databases.

\section{Analyse technique}
-jeu générique, doit marcher pour different set de personnage/portrait ...
-question imbriqué dans question relié par connecteur logique, récursif
-attribut peut avoir plusieurs valeurs


\section{Solutions technique}
-implementation d'un design pattern : Composit pattern

\section{Conclusion}
Not much of a paper, but it's a start.

\section{Annexe}

\begin{figure}
\caption{Composit pattern}
\centering
\includegraphics[scale=0.8]{Composite_UML_class_diagram.png}
\end{figure}

\section{Glossaire}
\textbf{design pattern} : 
un patron de conception (souvent appelé design pattern) est un arrangement caractéristique de modules, reconnu comme bonne pratique en réponse à un problème de conception d'un logiciel. Il décrit une solution standard, utilisable dans la conception de différents logiciels

\end{document}
